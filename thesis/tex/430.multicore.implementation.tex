%!TEX root = ../thesis.tex
\documentclass[../thesis]{subfiles}

\begin{document}
	\section{Implementation}
	Implementing the algorithms described in this chapter using C++ relied on heavy use of the Armadillo library. The MATLAB-like syntax made available by this library allowed for straight forward translation from the point method algorithms to the working code. See \cref{lst:multicore:diagonal:point} for a working example (stripped of header inclusion and namespace specification)

	\begin{listing}
		\inputminted[linenos,tabsize=4]{c++}{assets/code/multicore.diagonal.point.cpp}
		\caption[(Multicore) Point Diagonal Implementation]{Implementation of the matrix square root algorithm using the point method and the diagonal strategy.}
		\label{lst:multicore:diagonal:point}
	\end{listing}

	As for the block methods, the implementation requires a little more explanation. \texttt{span} is the type used by Armadillo to represent the ranges. Giving a range as a coordinate when accessing elements of the matrix actually retrieves a sub-matrix, which can then used as a matrix on its own. The BLAS calls, namely \texttt{GEMM}, are interfaced by the library using the standard multiplication operator (\texttt{*}) between two matrices. Lastly, the Sylvester equation is interfaced by the \texttt{syl()} function in Armadillo, which, after compilation, requires linkage to one of the many available LAPACK packages.

	The parallelization of the diagonal strategy implementations was done using OpenMP. Once again, the index system simplifies the process, allowing to parallelize the algorithm using only the \texttt{parallel for} directive.

\end{document}
