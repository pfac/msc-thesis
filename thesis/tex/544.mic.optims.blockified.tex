\documentclass[../thesis]{subfiles}

\begin{document}
	\subsection{Blocks as matrices}
	\label{subsec:mic:optims:blockified}

	After removing Armadillo, it becomes clear how \blas and \lapack calls access sub-matrices using the leading dimensions of the whole matrix. This simple approach is, however, error prone and better locality could be achieved were it not required to jump $n$ elements from one block column to the next.

	The matrices can be reorganized so each independent block is contiguous in memory, effectively making it an independent matrix. See \cref{eq:blockify} for example. $A$ is a regular column-major matrix, with the elements in the same column contiguous in memory, and each column also contiguous in memory. When trying to access the sub-matrix $A$ corresponding to the first two rows and columns, three elements of the first column must be skipped. In matrices where the dimension is large enough this translates into one memory access per block column. ``Blockifying'' $A$ generates $B$ where this does not happen because each block is now a column-major matrix, with all the blocks in the same column contiguous in memory, and the same being true for all the columns of blocks.

	\begin{equation}
		\begin{array}{ccc}
			\left[
			\begin{array}{c|c|c|c|c}
				 1 &  2 &  4 &  7 & 11  \\
				 0 &  3 &  5 &  8 & 12  \\
				 0 &  0 &  6 &  9 & 13  \\
				 0 &  0 &  0 & 10 & 14  \\
				 0 &  0 &  0 &  0 & 15  \\
			\end{array}
			\right] & \Rightarrow & \left[
			\begin{array}{cc|cc|c}
				 1 &  2 &  4 &  7 & 11  \\
				 0 &  3 &  5 &  8 & 12  \\
				 \hline
				 0 &  0 &  6 &  9 & 13  \\
				 0 &  0 &  0 & 10 & 14  \\
				 \hline
				 0 &  0 &  0 &  0 & 15  \\
			\end{array}
			\right] \\
			A & & B
		\end{array}
		\label{eq:blockify}
	\end{equation}

	This blockify operation does add some overhead to the initialization and to the cleanup (to revert the result to standard format), but it improves cache usage by through spatial locality. This overhead may or may not be worth depending on how good is this improvement and how it affects the computation.
\end{document}
