\documentclass[../thesis]{subfiles}

\selectlanguage{portuguese}
\begin{document}
	\chapter*{\abstractname}
		\section*{Computação Eficiente da Raíz Quadrada de uma Matriz em Plataformas Heterogéneas}

		Algoritmos de matrizes lidam regularmente com grandes quantidades de dados ao mesmo tempo, o que dificulta uma utilização eficiente da cache. Um trabalho recente de colaboração entre o Numerical Algorithms Group e a Universidade do Minho levou a uma abordagem por blocos para o algoritmo da raíz quadrada de uma matriz com melhorias de eficiência significativas, particularmente num ambiente multicore de memória partilhada.

		Arquitecturas de memória distribuída permaneceram inexploradas. Nestes sistemas os dados são distribuídos por diversos espaços de memória, incluindo aqueles associados a dispositivos aceleradores especializados, como GPUs. Sistemas com estes dispositivos são conhecidos como plataformas heterogéneas.

		Esta dissertação foca-se em estudar o algoritmo da raíz quadrada de uma matriz por blocos, primeiro num ambiente multicore e depois usando plataformas heterogéneas. Dois tipos de aceleradores são explorados: co-processadores \intel\xeonphi e GPUs \nvidia habilitados para CUDA.

		A implementação inicial confirmou as vantagens do método por blocos e mostrou uma escalabilidade excelente num ambiente multicore. A mesma implementação foi ainda usada para o \intel\xeonphi, mas os resultados de performance obtidos ficaram aquém do comportamento esperado e da alternativa usando apenas CPUs. Várias optimizações foram aplicadas a esta implementação comum, conseguindo reduzir a diferença entre os dois ambientes.

		A implementação para dispositivos CUDA seguiu um modelo de programação diferente e não pôde beneficiar the nenhuma das soluções anteriores. Também exigiu a implementação de rotinas BLAS e LAPACK, já que nenhum dos pacotes existentes se adequa aos requisitos desta implementação. A performance medida também mostrou que a alternativa usando apenas CPUs ainda é a mais rápida.
\end{document}
