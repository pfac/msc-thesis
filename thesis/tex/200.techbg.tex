\documentclass[../thesis]{subfiles}

\begin{document}
	\chapter{Technological Background}
	For more than half a century, computational systems have evolved at an increasing rate, fueled by a similar demand in computing power. The invention of the digital transistor, in 1947, smaller and faster than its predecessor, made computers smaller and more power efficient. Integrated circuits further reduced the space and power requirements of computers, which in turn led to the emergence of the microprocessor. The evolution of the microprocessor in the following decades followed had two main contributors: the number of transistors and their clock frequency.

	A processor's clock frequency is the rate at which it can issue instructions, which means that a higher frequency roughly translated in more instructions being processed each second. Increasing a processor's frequency had its impact on software development. While each generation of processors offered a faster clock frequency, programs would simply run faster without any change to the underlying source code. On the other hand, increasing the number of transistors in a chip allowed to create complex systems to optimize software execution. These focused in two critical performance techniques: the exploration of \ilp and the use of memory caches.

	In 1965, Gordon Moore, in an attempt to predict the evolution of integrated circuits during the following decade projected that the number of transistors would double every year \cite{Moore:1965}. In 1975, adding more recent data, he slowed the future rate of increase in complexity to what is still today known as Moore's Law: the number of transistors in a chip doubles every two years \cite{Moore:1975,ComputerHistory:Moore}.

	In 2003, the evolution of the microprocessor reached a milestone. The increase in clock frequency of microprocessors closely followed the increase in the number of transistors, but the power density in processors was approaching the physical limitations of silicon with air cooling techniques. Continuing to increase the frequency would require new and prohibitively expensive cooling solutions. With Moore's Law still in effect, and the lack of more \ilp to explore efficiently, most vendors now focus on creating symmetric multi-core processors for the mass market.

	\tdi{The evolution of the computing technologies}
		\section{Parallelism}
		\section{Heterogeneous Platforms}
		\section{Development Tools}
\end{document}
