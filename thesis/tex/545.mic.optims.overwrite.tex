\documentclass[../thesis]{subfiles}

\begin{document}
	\subsection{Overwrite}
	\label{subsec:mic:optims:overwrite}
	All the implementations so far assume at least two distinct matrices are used in the algorithm, one for $T$ and another one for $U$. Aside from dependencies, this implies that when a block $U_{ij}$ is being computed, another block $T_{ij}$ must also be present. The memory footprint becomes even larger with the \usb format where the conversion generates a second re-organized matrix, which is then used as the input matrix for the algorithm. The algorithm computes a third matrix with the result, also in \usb format, which then has to be re-organized into a fourth matrix with the final result in the standard format.

	\blas and \lapack routines minimize the memory footprint by overwriting one of the operands with the result of the operation. In the case of the matrix square root algorithm this is also possible since only one element/block in the input matrix $T$ is used in the computation of each element/block in $U$. Consequently, the memory footprint can be easily reduced by overwriting $T$ with $U$.
\end{document}
