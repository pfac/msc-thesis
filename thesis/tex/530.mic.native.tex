\documentclass[../thesis]{subfiles}

\begin{document}
	\section{Native execution}
	\label{sec:mic:native}

	\tdg{Native mode advantages}
	The native execution mode provides several advantages over its alternatives. To start, it makes one extra core available. Given that the coprocessor's cores architecture is based on the x86 \isa, it also reduces the development time considerably, as a \cpu functional implementation requires only to be rebuilt targeting the \mic architecture in order for the device to be able to run it natively. It also skips the communication necessary in an offload-based implementation, which is a potential bottleneck for many applications.

	\tdg{self-explanatory}
	For these reasons, this mode was selected as a first attempt to use the \intel\xeonphi Coprocessor.

	\tdg{build system had to be changed}
	As previously stated, no change was required to the code developed in the previous chapter, the only change being in the build process (the \texttt{-mmic} flag). Yet, the previous build system was not prepared for the \intel\xeonphi Coprocessor, since it implied cross-compilation, and had to be adapted.

	\subfile{tex/531.mic.native.results.tex}
\end{document}
